\section{Ziele}
Der dynamische Austausch der Daten oder der KI soll durch eine Schnittstelle zwischen Datenerhebung und Datenverarbeitung gelöst werden. Die Betreiber der KI sollen lediglich eine Konfigurationsdatei anpassen müssen, um die Daten für die KI vorzubereiten. Ziel der Bachelorarbeit wird es sein, ein Programm zu entwickeln, welches mithilfe dieser Konfigurationsdatei Daten aus einem Datenerhebungsalgorithmus mittels einer REST-API Schnittstelle annimmt und diese automatisch für eine vorgesehene KI präpariert. Im Anschluss soll das Programm mit den vorbereiteten Daten eine Anfrage an die KI senden und das Ergebnis dem Benutzer wieder zurücksenden. 

Wichtig bei der Entwicklung ist es, dass sowohl Datenerhebung als auch Datenverarbeitung modular entwickelt werden. Da die Schnittstelle als REST-API entwickelt wird, können zukünftige Entwickler die Schnittstelle nutzen, auch wenn sich die Daten oder die KI's verändern sollten. 

Die REST-API wird mithilfe des Python Web-Frameworks Flask entwickelt. Dies bietet eine einfache und schnelle Möglichkeit einen Backend-Server aufzusetzen. Des weiteren ist Python eine umfangreiche und einfach zu benutzende Programmiersprache für Datenverarbeitung. 

Die Konfiguration der Datenverarbeitung wird mit dem Web-Frontend Framework Angular entwickelt. Dort soll der Entwickler eine Schritt für Schritt Anleitung für die Konfiguration der Schnittstelle präsentiert bekommen. Dort kann die Art der Daten, deren Format und die anzuschließende KI ausgewählt werden. 

Mithilfe dieser Schnittstelle soll sich der Arbeitsaufwand verringern, der durch die Anbindung neuer KI's entsteht. Damit können erfahrene Entwickler entlastet werden und Personalkosten gespart werden. Ebenso können schnell und ohne großen Mehraufwand alternative Aufbereitungen der Daten oder neue KI's getestet werden.