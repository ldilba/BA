\section{Grundlagen}

\subsection{Python API mit Flask}
Python ist eine um 1991 von Guido van Rossum entwickelte Programmiersprache. Bei der Entwicklung von Python wurde ein besonderer Fokus auf die Lesbarkeit von Code gesetzt. Dank der simplifizierten Syntax im Vergleich zu anderen höheren Programmiersprachen wie Java oder C\#, ist Python auch in Bereichen, wie in der Mathematik oder der Wissenschaft ein häufig genutztes Werkzeug. Python bietet ebenfalls die Möglichkeit, von anderen Entwicklern bereitgestellte Bibliotheken in das eigene Projekt zu integrieren.\footcite{python2021python} 

Flask ist eine der verfügbaren Bibliotheken, die ein Framework für die Implementierung eines Webbasierten \ac{api} bereitstellt. Eine API dient dazu, Funktionen und Routen zu definieren, um die Kommunikation zwischen dem Frontend und dem Backend herzustellen. Das Flask Framework ist im Gegensatz zu anderen Frameworks sehr klein. Dies ermöglicht ein schnelles aufsetzen und entwickeln. Da Flask nur die nötigsten Grundlagen für eine API mitliefert, ist der Code besser lesbar und damit für andere Entwickler besser wartbar.\footcite{grinberg2018flask} 

Die Flask API wird für die Anbindung des Frontends an die Datenbank, sowie die Anbindung an die Kommunikationsschnittstelle von RabbitMQ verwendet. Sie nimmt die Daten oder die Eingaben des Nutzers entgegen und vermittelt sie an den richtigen Dienst, damit sie von einer KI-Schnittstelle ausgewertet werden können. Anschließend kann die API angefragt werden, ob es bereits Antworten von einer \ac{ki} zu der vorher geschickten Anfrage gab. Falls die API die Auswertung der KI erhalten hat, wird diese ans Frontend geschickt, um sie dort anzeigen zu können.  

\subsection{Angular Frontend}
Eine grundlegende Website wird klassisch mit \ac{html} und JavaScript erstellt. Um eine moderne Website zu entwickeln, die ihren Inhalt nicht beim ersten Aufrufen lädt, sondern erst dann, wenn er benötigt wird, müssen Konzepte wie \ac{ajax} verwendet werden. Angular ist ein von Google gebaut und gepflegtes Open Source Framework, welches das Entwickeln von komplexen webbasierten Anwendungen vereinfachen soll. Angular bietet im Gegensatz zu anderen Webframeworks wie React und Vue.js eine vollumfängliche Bibliothek, mit der nahezu alle Aspekte in der Web Entwicklung abgedeckt werden können.\footcite{moiseev2018angular}

In Angular wird die Programmiersprache TypeScript verwendet. Diese ist eine Erweiterung der Programmiersprache JavaScript und implementiert Konzepte wie feste Typisierung von Variablen. Weitere Konzepte wie Dependency Injection oder die Trennung von \ac{bl} und \ac{ui} ermöglichen eine schnelle Entwicklung von komplexen Systemen. 

Das Frontend wird für die Ein- und Ausgabe der Daten verwendet. Der Nutzer kann auf der Webseite seine Suchanfrage in ein Textfeld schreiben und anschließend auf den Server hochladen. Im nächsten Schritt wird die Möglichkeit bereitgestellt, die eingegeben Daten automatisiert zu bearbeiten und zu manipulieren. Im gleichen Zug wird die Eingabe des Nutzers in ein für die KI verständliches Format konvertiert. Im letzten Schritt kann der Nutzer die Anfrage an das Backend schicken, dass mit der Analyse der Eingabe begonnen werden soll. Das Frontend fängt daraufhin an beim Backend in regelmäßigen Abständen nach Antworten der KI zu fragen. Wenn Antworten vorhanden sind, können diese in einer Liste visualisiert werden.

\subsection{Redis API Cache}
Redis ist eine In-Memory Key-Value Datenbank. Im Gegensatz zu \ac{rdbms} wie MySQL oder PostgreSQL werden in Redis keine festen Tabellenstrukturen hinterlegt. Redis gehört damit zur Kategorie der NoSQL Datenbanken (Not Only SQL). Key-Value Stores sind kein Ersatz für eine relationale Datenbank, bieten aber für bestimmte Bereiche große Vorteile. Durch das Fehlen von komplexen Strukturen innerhalb der Datenbank, kann Redis Anfragen weitaus schneller als andere Datenbanksysteme bearbeiten. Da Redis im \ac{ram} ausgeführt wird,  werden die Daten grundsätzlich nicht persistent gespeichert. ACID (Atomicity, Consistency, Durability and Isolation) Konformität wird mit Redis ebenfalls nicht gewährleistet. Für den Einsatzzweck als Cache in einer Cloud Umgebung ist Redis allerdings sehr gut geeignet. \footcite{paksula2010persisting}

Innerhalb des Redis Key-Value Stores werden alle relevanten Daten gespeichert, die ein Nutzer während seiner Benutzung der Software produziert. Dort werden ebenfalls die Zwischenergebnisse abgespeichert, die die KI während der Analyse erstellt.

\subsection{MySQL Datenbank für Services und Logs}
MySQL ist ein um 1995 erschienenes Open-Source RDBMS. MySQL ist eines weitverbreitetsten und schnellsten Datenbanksysteme in seiner Kategorie. \footcite{dubois2008mysql}

In relationalen Datenbanken werden Daten strukturiert in Tabellenform abgespeichert. Einzelne Tabellen können Verlinkungen und Referenzen auf andere Tabellen haben, damit die Zusammengehörigkeit der Daten beschrieben werden kann, ohne Daten redundant speichern zu müssen. In MySQL, wie auch anderen RDBMS, werden Tabellenstrukturen und Daten persistent abgespeichert. In-Memory Datenbanken wie Redis können Daten über Umwege auch persistent speichern, jedoch müssen dafür größere Anpassung an der Konfiguration von Redis vorgenommen werden.

Das RDBMS MySQL wird unter Anderem für die Speicherung der Logs, die der Flask Server während der Verarbeitung von Requests oder Nachrichten an die KI produziert, verwendet. Ein weiterer Einsatzzweck der MySQL Datenbank ist die Speicherung der im System registrieren KI-Services. Ein Dienst kann über die Flask API im System registriert oder deregistriert werden. Das Frontend kann sich im Anschluss eine Auflistung der verfügbaren Services vom Backend ziehen.

\subsection{Kommunikation mit RabbitMQ}
Damit eine Kommunikation zwischen unabhängigen Programmen möglich wird, muss es einen Zwischendienst geben, der die Nachrichten von von Programm zum anderen transportiert. Beider Kommunikation zwischen einer Website und einer API wird das \ac{http} verwendet. Dieses stellt sicher, dass die Information, ob die Nachrichten am anderen Ende angekommen sind, vorhanden sind. Sollte eine Nachricht nicht angekommen sein, hat der Absender die Möglichkeit die Nachricht erneut zu schicken. Problematisch wird diese Herangehensweise, wenn die Antwortzeit sehr lang wird oder ungewiss ist, ob überhaupt eine Antwort kommen wird. 

RabbitMQ ist ein eine nachrichtenorientierte Middleware, die die Kommunikation zwischen zwei oder mehreren Programmen durch das  \ac{amqp} ermöglicht. Im Gegensatz zu einer direkten Kommunikation zwischen Client und Server wie bei HTTP, wird in RabbitMQ eine Queue implementiert, in der alle Anfragen gesammelt werden. Jeder Client kann Nachrichten in die Queue reinschreiben. Diese Nachrichten werden dort so lange gespeichert, bis sie von einem Dienst ausgelesen werden. Durch diese Herangehensweise wird eine asynchrone Kommunikation zwischen Client und Server ermöglicht. Da RabbitMQ frei von den Handshakes des HTTP ist, sind die Schreib- und Lesezeiten deutlich schneller.\footcite{ionescu2015analysis}

Die Middleware RabbitMQ wird für die Kommunikation zwischen der Flask API und den KI-Services genutzt. Der im Frontend vom Nutzer eingegebene Text-Input wird an die Flask API geschickt. Die Flask API modifiziert den Text im Anschluss so, dass es mittels der \ac{json} über den RabbitMQ Service in die Queue geschrieben werden kann. Jeder KI-Service hat eine Queue einprogrammiert, aus der die Nachrichten ausgelesen werden. Diese Nachrichten können dann verarbeitet und im Anschluss in eine Response-Queue geschrieben werden. Das Flask Backend kann diese Response-Queue auslesen und die einzelnen Antworten dann zusammenbauen.

\subsection{KI-Service}
Die KI-Services sind alleinstehende Programme, die die Aufgabe haben, Nachrichten anzunehmen, sie zu transformieren, zu analysieren und anschließend ein oder mehrere Ergebnisse zurückzugeben. 

Um die Nachrichten empfangen und die Ergebnisse zurücksenden zu können, muss in jedem Service eine AMQP Verbindung zu RabbitMQ hergestellt werden.

%In dem im Prototypen implementierten Service zur Textähnlichkeitssuche 

\subsection{Logs visualisieren in Grafana}
text

\subsection{Deployment mit Docker}
text
