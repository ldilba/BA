\section{Evaluation}
Die Evaluation der Software wurde mit einem Mitarbeiter von CONET während der Entwicklungsphase durchgeführt. Der Prototyp wurde in einem Zeitraum von sechs Wochen implementiert. Die Durchführung eines der Code-Reviews fand in regelmäßigen Abständen statt. Während des Code-Reviews wurde der derzeitige Stand der Software präsentiert, bestehende Probleme und Schwachstellen besprochen und die weiteren Entwicklungsschritte bis zum nächsten Review geplant.

\begin{itemize}[leftmargin=6em]
\item [Woche 1:] Recherche zu den Bereichen Microservice-Architekturen und Schnittstellen war die hauptsächliche Aufgabe zu Beginn des Projekts. Im ersten Code-Review wurde die geplante Architektur vorgestellt. Das primäre Feedback zur geplanten Architektur war, nicht nur eine allgemeingültige Schnittstelle zu entwickeln, sondern die Implementierung einer Textähnlichkeitssuche in den Vordergrund zu stellen. Anhand dieser KI soll die Funktionalität der Schnittstelle gezeigt werden.
\item [Woche 2:] Das Konzept für die Implementierung der Schnittstelle wurde nach einer Anpassung der geplanten Architektur entworfen. Dieses Konzept umfasst ein Architekturdiagramm, welches die Komponenten innerhalb der Software darstellt, ein Prozessablaufdiagramm, in dem die Funktionsweise und Abläufe der Schnittelle visualisiert sind und Mockups für die zu implementierende Website. Die ursprünglich geplante Kommunikation von dem Service zur API sollte ebenfalls mit RabbitMQ umgesetzt werden, statt den API-Endpunkt des Backendes zu nutzen. Dies vereinheitlicht den Programmcode und beschleunigt die Antwortzeit der KI-Services.
\item [Woche 3:] Der Großteil der REST-API wurde implementiert. Innerhalb dieses Code-Reviews wurde auf den Aufbau der Routen, sowie die Anbindung an die MySQL-Datenbank und den Redis-Cache geachtet. Hierzu gab es seitens CONET keine Kritik oder Verbesserungsvorschläge.
\item [Woche 4:] Die API wurde fertiggestellt und anschließend im Code-Review vorgestellt. Besonderer Fokus lag dabei auf der Implementierung des Errormanagement- und Loggingsystems. Innerhalb dieses Reviews wurde auch Grafana zur Visualisierung der Logs vorgestellt. Ein Kritikpunkt an der gewählten Kombination aus einer MySQL-Datenbank und Grafana war, dass das System bei der Anzeige einer größeren Anzahl an Logs langsam wird. Die Anzahl der angezeigten Logs in Grafana musste demnach beschränkt werden. Des Weiteren benötigt das System zum Error-Handling für den Einsatz in einer Produktivumgebung eine genauere Aufteilung der Error-Codes. Im aktuellen Zustand des Prototyps wird bei einer fehlerhaften Eingabe der Error-Code 500 zurückgegeben. Dies ist für den realen Einsatz der Anwendung allerdings nicht ausreichend.
\item [Woche 5:] Die Website wurde entwickelt und an die API angeschlossen. Die visuelle Gestaltung Der Website basierte auf den in Woche 2 erstellten Mockups. Die Nutzung der Website war nach Einschätzung von CONET erst mit einer kleinen Einführung verständlich. Die Website, für sich alleinstehend, bietet zu wenig Erklärungen oder Hilfestellungen, als das sie ein Unbeteiligter nutzen könnte. Die Funktionalität der Website ist jedoch gewährleistet und deckt die in den Anforderungen erhobenen Punkte ab.
\item [Woche 6:] Der KI-Service zur Textähnlichkeitssuche wurde implementiert. Zur Kommunikation zum Backend wurde der Service an den Dienst RabbitMQ angeschlossen. Anschließend wurden die Teile der Software über Docker in einzelne Container verpackt. Dadurch konnte die Softwarearchitektur auf den Rechnern mehrerer CONET-Mitarbeiter installiert und getestet werden.
\end{itemize}

Nach dem Abschluss der Code-Reviews wurde der entwickelte Prototyp vor mehreren Mitarbeitern von CONET präsentiert. Im Anschluss an die Präsentation wurde das Ergebnis der Bachelorarbeit mit den zu Beginn erhobenen Anforderungen verglichen. Grundsätzlich erfüllt der Prototyp sämtliche Anforderungen. Die entworfene Architektur wurde für modernen, performant und skalierbar befunden. Sie bietet eine gute Grundlage, um komplexere Systeme darauf aufbauen zu können.

Es gibt, wie auch in den Code-Reviews angemerkt wurde, einige Bereiche mit Erweiterungspotential gibt. Diese betreffen jedoch ausschließlich die visuelle Aufbereitung und Nutzerinteraktion mit dem System. Die angemerkten Kritikpunkte sind folgende:

\begin{itemize}
\item Die Nutzung der Website ist nicht intuitiv für neue Nutzer
\item Grafana wird mit MySQL zu langsam für eine große Anzahl an Logs
\item Der Nutzer kann falsche Eingaben auf der Website tätigen, ohne darüber in Kenntnis gesetzt zu werden
\item Die Installation der Software ist relativ zeitaufwändig, wenn Docker nicht bereits installiert ist
\item Die Elasticsearch-Datenbank im KI-Service kann nicht mit weiteren Daten befüllt werden
\end{itemize}
  