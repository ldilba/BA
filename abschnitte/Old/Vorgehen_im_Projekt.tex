\section{Vorgehen im Projekt}
\textit{(Autoren: Laurenz Dilba und Ann-Sophie Grünwald)}

In diesem Kapitel wird das Vorgehen im Projekt anhand einiger Beispiele erläutert. Zunächst wird auf die Projektorganisation und die Organisationsstruktur eingegangen, um im Folgenden die genaue Umsetzung des Projektes zu beschreiben. 

\subsection{Einleitung in das agile Projektmanagement}
\subsubsection{Allgemeine Definition von agilem Projektmanagement}
Um den Begriff des agilen Projektmanagements beschreiben zu können, müssen zunächst die Begriffe \glqq Projektmanagement\grqq{} und \glqq Agilität\grqq{} definiert werden. 

Bei einem Projekt handelt es sich um ein zeitliches Unterfangen, mit dem Ziel ein einzigartiges Ergebnis zu entwickeln. \footcite[Seite 9]{Brandstäter2013} %Brandstäter2013
Projektmanagement beinhaltet die Anwendung von Wissen, Können, Werkzeugen und Techniken, um Projektanforderungen zu erfüllen. \footcite[Seite 9]{Brandstäter2013}
Agilität ist die Fähigkeit, in einer sich stets verändernden Unternehmensumwelt auf Veränderungen reagieren zu können. \footcite[Seite 9]{Brandstäter2013}

Ziel eines agilen Projekts ist es, ein Ergebnis in einer sich verändernden Umwelt liefern zu können. Dazu findet zum Beispiel ein konstanter Austausch mit den Personen, die am Projekt interessiert sind und eventuell Einfluss haben, welche als Stakeholder bezeichnet werden, statt.\footcite[Seite 9]{Brandstäter2013} Dadurch ist es dem Entwicklerteam möglich, dauerhaft über alle Anforderungen auf dem aktuellen Stand zu sein.

Agile Projekte werden nach den Grundsätzen des \textit{Agilen Manifests} geplant und durchgeführt. Dieses wurde von verschiedenen Autoren entworfen, um eine allgemeine Definition von Agilität zu bestimmen. Es beinhaltet verschiedene Werte und Prinzipien, die als Leitsatz für die agile Arbeit dienen. \footcite[Seite 9]{Brandstäter2013}

\subsubsection{Allgemeine Definition von Scrum}
Scrum stellt ein Framework zur Entwicklung komplexer Produkte dar. Dieses bietet ein Projektmanagementsystem, das es den Entwicklern ermöglicht, während der Auslieferung eines Produktes auf Veränderungen regieren  und sich an unvorhersehbare Ereignisse anpassen zu können. Es bietet dem Entwicklerteam genug Freiraum, selbständig und adaptiv auf Veränderungen in der Umwelt zu reagieren.\footcite[Seite 11]{Brandstäter2013}

Scrum ist ein iterativer Prozess, bei dem jede Iteration als Sprint bezeichnet wird. Ein Sprint ist ein fester Zeitraum, mit einem innerhalb des Teams vereinbarten Ziel, welches als Sprintziel bezeichnet wird, in dem die Entwickler eigenständig und ohne Einfluss von außen am Produkt arbeiten. Dieser Prozess hat keine allgemein festgelegte Länge. Üblicherweise beträgt sie jedoch nicht länger als 30 Tage. Resultat des Sprintzykluses sollte ein funktionsfähiges Produktinkrement sein. \footcite[Seite 13]{Brandstäter2013} Während eines Sprints finden regelmäßig Meetings innerhalb des Entwicklerteams statt. Ziel ist es, den Fortschritt des Teams aufzuzeigen und das weitere Vorgehen, Probleme und Hindernisse anzusprechen.\footcite[Seite 14]{Brandstäter2013}

Während des Sprints werden Aufgaben abgearbeitet, welche in Form von User Stories festgehalten werden. Eine User Story besteht aus einer Beschreibung, weiterführenden Details und Akzeptanzkriterien.

\subsubsection{Vorteile von Scrum}
\begin{itemize}
\item Übersicht über alle Aufgaben\vspace{0.1cm}\\ 
Durch eine einheitliche Aufgabensammlung, die als Backlog bezeichnet wird, können die Anforderungen an das Projekt, welche in Form von User Stories formuliert sind, festgehalten und verfolgt werden. Somit ist der aktuelle Entwicklungsstand des Projekts für jeden jederzeit einsehbar. \vspace{0.1cm}

\item Klare Zuständigkeiten\vspace{0.1cm}\\ 
Jeder Entwickler hat die Möglichkeit sich selbstständig Aufgaben zuzuweisen, für deren Abschluss er die Verantwortung übernimmt. Er ist außerdem Ansprechpartner bei Rückfragen bezüglich der zugewiesenen User Story. \vspace{0.1cm}

\item Gemanagte Zeiteinteilung\vspace{0.1cm}\\ 
Durch die Einschätzung der geplanten idealen Arbeitsstunden der Entwickler und des benötigten Zeitaufwands aller Aufgaben, kann der Aufgabenumfang pro Sprint realistisch geplant werden.\vspace{0.1cm}

\item Paralleles Arbeiten\vspace{0.1cm}\\ 
User Stories werden während der Anforderungserhebung so formuliert, dass sie unabhängig voneinander bearbeitet werden können.\vspace{0.1cm}
\item Funktionierender Prototyp\vspace{0.1cm}\\ 
Durch den iterativen Entwicklungsprozess, steht am Ende jedes Sprints ein funktionierender Prototyp zur Verfügung, den sowohl Kunden als auch andere Stakeholder beurteilen können. 

\end{itemize}

\subsection{Einführung von Scrum im Projekt}
\subsubsection{Zeitliche Planung}
Zu Beginn der Planungsphase wurden vier Sprintzeiträume festgelegt. Das Projekt beinhaltete drei Sprints á drei Wochen und einen Sprint á zwei Wochen. 
\begin{enumerate}
\item Sprint: 25.10.2021 - 14.11.2021
\item Sprint: 15.11.2021 - 05.12.2021
\item Sprint: 06.12.2021 - 23.12.2021
\item Sprint: 10.01.2022 - 21.01.2022
\end{enumerate}

\subsubsection{Planung der Kapazität}
Zu Beginn der Planungsphase wurde die Sprint Velocity geschätzt. Diese gibt an wie viele Story Points alle Entwickler zusammen innerhalb eines Sprints bearbeiten können. Ein Story Point stellt eine ideale Arbeitsstunde dar. Ideal ist eine Arbeitsstunde dann, wenn der Entwickler ohne Ablenkung, Unterbrechung oder größere Recherche arbeiten kann.

An dem Projekt OK Golf waren zehn Entwickler beteiligt. Für jeden Entwickler wurde in einem anfänglichen Meeting eine wöchentliche Kapazität von sechs Arbeitsstunden eingeplant. 
Für die ersten drei Sprints ergaben sich daraus 180 Arbeitsstunden pro Sprint. Um diese in ideale Arbeitsstunden umzurechnen, wurde ein Puffer von 20 Arbeitsstunden abgezogen. Daraus ergaben sich 160 Story Points pro Sprint.
Für den letzten Sprint wurden aufgrund der verkürzten Sprintlänge 100 Story Points angesetzt.

\subsubsection{Ermittlung der Anforderungen an das Projekt}
In Absprache mit der Objektkultur wurden allgemeine Anforderungen an die Software erhoben, aus denen User Stories abgeleitet werden konnten. Die Liste der erstellten User Stories wird als Backlog bezeichnet. Der Aufwand jeder User Story wurde im Anschluss in einem Scrum Poker Meeting mit dem gesamten Team bestimmt und in Form von Story Points festgehalten. 

Mittels der Sprint Velocity und den Story Points der jeweiligen User Story, kann abgeschätzt werden, wie viele Aufgaben in einem Sprint erledigt werden können. Da nicht jede Story die gleiche Gewichtung hat, wurde eine Abstufung in die Prioritätsgrade \glqq Must\grqq{}, \glqq Could\grqq{} und \glqq Wish\grqq{} vorgenommen. Ebenso wurden die Aufgaben in Frontend und Backend mittels den Tags \glqq Flutter\grqq{} und \glqq API\grqq{} unterteilt, um die Übersicht für die Entwickler zu verbessern.

\subsection{Ablauf der Sprintzyklen}

\begin{figure}[H]
  \centering
    \includegraphics[width = 15cm]{bilder/Sprintzyklus}
    \caption{Ablauf eines Sprints}
\end{figure}

In Abbildung 1 ist ein beispielhafter Ablauf eines Sprints abgebildet. Planungstermine werden als abgerundetes Viereck und allgemeine Besprechungstermine in Form eines Kreises dargestellt. Die Fixpunkte des Sprints, der Sprintstart und das Sprintende, werden durch ein Dreieck visualisiert.

\subsubsection{Product Backlog}
Vor dem Sprintstart haben die Projektleiter den aktuellen Stand des Product Backlogs bewertet. Besonderer Fokus lag hierbei auf der Aktualität der User Stories. Aufgaben, die sich verändert haben oder weggefallen sind, wurden in diesem Schritt angepasst und die zugewiesenen Story Points erneut evaluiert. 
\subsubsection{Sprint Backlog}
Aus dem überarbeiteten Product Backlog wurden im Anschluss die zu bearbeitenden User Stories für den kommenden Sprint ausgewählt. Hierbei wurde darauf geachtet, dass die aufsummierten Story Points der einzelnen User Stories nicht die zu Beginn des Projektes geplante Sprint Velocity überschreiten. Ebenso wurde in diesem Schritt, in Absprache mit dem Entwicklerteam, ein Sprintziel festgelegt. 
\subsubsection{Sprintstart}
Im Anschluss konnte der Sprint gestartet werden. Dieser umfasste einen Zeitraum von drei Wochen mit Ausnahme des letzten Sprints, welcher aufgrund der Projektdauer auf zwei Wochen reduziert wurde. Innerhalb dieses Zeitraums wiesen sich die Entwickler selbstständig die im Sprint Backlog geplanten User Stories zu und arbeiteten diese ab. 
\subsubsection{Jour Fixe und Absprache mit dem Kunden}
Um bisher erreichte Fortschritte und eventuelle Probleme besprechen zu können, wurde in den ersten zwei Wochen des Sprints Freitags um 13:30 Uhr ein Jour Fixe abgehalten. Der letzte Freitag des Sprints wurde dazu genutzt, sich mit dem Kunden, der ObjektKultur, auszutauschen. Inhalt dieses Meetings war die Präsentation des aktuellen Stands der mit Flutter entwickelten App, ein Überblick über die User Stories im Backlog und der Entwicklungsstand sowie die Dokumentation der API. Ebenso wurde über die technische Umsetzung der Azure Cognitive Services KI diskutiert. 
\subsubsection{Sprintende und Sprint Review}
Am Sprintende sollten idealerweise alle geplanten User Stories abgearbeitet sein. Der Hauptfokus lag allerdings auf der Erreichung des zuvor gesteckten Sprintziels. 

In dem anschließenden Sprint Review, welches die Projektleiter nach dem Ende des zweiten Sprints eingeführt haben, wurden alle nicht erledigten User Stories aus dem vergangenen Sprint innerhalb eines Meetings mit dem Entwicklerteam besprochen und eventuelle Problemfelder identifiziert. Erarbeitete Lösungsansätze für die erkannten Probleme wurden im darauffolgenden Sprint sowohl bei der Planung als auch bei der Umsetzung berücksichtigt. Der Grund für die Einführung des Sprint Reviews war die mangelnde Transparenz über den Erfolg des Sprints.  

\subsection{Organisatorische Hilfsmittel}
\subsubsection{Sprintplanung und Sprintdurchführung}
Zur Aufgabenplanung, Erstellung des Sprint Backlogs und der Durchführung des Sprints wurde die Software Azure DevOps verwendet, welche der Kunde zu Beginn des Projektes bereitstellte. 
\begin{figure}[H]
  \centering
    \includegraphics[width = 12cm]{bilder/Burndownchart}
    \caption{Burndownchart des 3. Sprints, Stand 17.12.2021}
\end{figure}
Azure DevOps bietet zudem Analysemethoden in Form von Burndown Charts, wie in Abbildung 2 dargestellt. Diese Diagramme bieten die Möglichkeit den Fortschritt des aktuellen Sprints zu verfolgen. Die graue, diagonale Linie stellt den idealen Abarbeitungsverlauf dar. Die blaue Fläche visualisiert den tatsächlichen Arbeitsverlauf. Sollte hier bereits ein Trend erkennbar sein, dass nicht alle Aufgaben abgearbeitet werden können, besteht die Möglichkeit frühzeitig Maßnahmen zu ergreifen. 
\subsubsection{Kommunikation im Team}
Die Hauptkommunikationskanäle innerhalb des Projekts waren WhatsApp und Microsoft Teams.

Der Messengerdienst WhatsApp diente hauptsächlich dem Schriftverkehr. Zum einen war die App bereits vor dem Projekt auf jedem mobilen Endgerät im Team installiert, zum anderen konnte durch Push-Benachrichtigungen eine hohe Erreichbarkeit gewährleistet werden. 

Die Kollaborationssoftware Microsoft Teams bietet die Möglichkeit Videokonferenzen abzuhalten. Dieses Feature wurde für die wöchentlichen Besprechungstermine innerhalb des Teams wie auch für die Kommunikation mit dem Kunden genutzt. Microsoft Teams beinhaltet zusätzlich dazu eine Cloud-Speicherlösung für die Dokumentenverwaltung des Teams.

\subsubsection{Kommunikation mit dem Kunden}
Um stets sicherzustellen, dass sich die bearbeiteten Aufgaben mit den Anforderungen der Objektkultur decken, wurden regelmäßige Online-Besprechungstermine vereinbart. Diese begannen in einem Drei-Wochen-Zyklus. Da ein häufigerer Austausch auch außerhalb der Sprinttermine für sinnvoll erachtet wurde, wurde der Zyklus auf zwei Wochen verkürzt. Um den Kunden von den Terminen in Kenntnis zu setzen, wurde wenige Tage vorher eine Agenda für den kommenden Termin per E-Mail verschickt. 

Folgende Termine fanden innerhalb des Projektzeitraums statt:

\begin{itemize}
\item 01.10.2021 Kennenlernen
\item 08.10.2021 Vorstellung des Projekts, Planung und Terminabsprache
\item 22.10.2021 Präsentation Backlog und Diskussion der User Stories
\item 12.11.2021 Vorstellung der Ergebnisse des ersten Sprints
\item 26.11.2021 Vorstellung des Entwicklungsstands
\item 10.12.2021 Zwischenpräsentation
\item 07.01.2021 Vorstellung des Entwicklungsstands
\item 21.01.2021 Endpräsentation
\end{itemize}


\subsection{Aufgabenverteilung der Teammitglieder}
Das aus zehn Mitgliedern bestehende Projektteam hat sich in vier Arbeitsgruppen eingeteilt. Die Mitglieder der Planungs- und Organisationsgruppe haben zusätzlich die anderen Gruppen unterstützt.
Im folgenden sind die Gruppen und deren Mitglieder aufgelistet.

\begin{enumerate}
\item Planungs- und Organisationsgruppe
\begin{itemize}
\item Laurenz Dilba
\item Ann-Sophie Grünwald
\end{itemize}
\item Frontend-Entwicklungsgruppe
\begin{itemize}
\item Robert Diehl
\item Ann-Sophie Grünwald
\item Christian Hendricks
\item Nico Jülich
\end{itemize}
\item Backend-Entwicklungsgruppe
\begin{itemize}
\item Mario Höhnighausen
\item Alexander Tewes
\item Fabian Weiß
\end{itemize}
\item KI-Entwicklungsgruppe
\begin{itemize}
\item Ron Autenrieb
\item Laurenz Dilba
\item Viktor Schander
\end{itemize}
\end{enumerate}
