\section{Bezug zum bisherigen Studium}
Da das Projekt DashNET und insbesondere das Backend des Projektes komplett ohne Vorgaben, im Bezug auf den Entwurf der Softwarearchitektur entwickelt werden konnte, waren mehrere Lehrveranstaltungen äußerst hilfreich. Den größten Einfluss hatte das Wirtschaftsinformatik-Projekt bei Prof. Dr. Matthias Bertram. Dort hatten wir die Möglichkeit eine Software entwickeln, die wir selber geplant und entworfen haben. Dabei haben wir uns in Themengebiete wie REST Spezifikationen, APIs im Allgemeinen, JSON Dateien, HTTP Requests sowie viele weitere Themengebiete eingearbeitet. Dies hatte einen sehr großen Mehrwert für die Entwicklung der Python API für das DashNET Projekt. 

Ebenfalls wertvoll für das Projekt waren die Veranstaltungen Software Engineering 1 und Software Engineering 2. Dort wurde uns vor allem das Projektmanagement beigebracht, aber auch der Umgang mit Versionskontrollsoftware wie Git, die bei dem DashNET und dem ePEP Projekt zum Einsatz kam. In diesen Veranstaltungen wurden die Konzepte des Model-View-Controller Patterns vermittelt. Dieses Pattern war in der ePEP Software wiederzufinden.  

Zuletzt hatten auch die Veranstaltungen Programmieren 2 und Statistik Einfluss auf das Praktikum. In diesen Veranstaltungen gab es die ersten Berührungspunkte mit Python und fortgeschrittenen Programmiertechniken.

