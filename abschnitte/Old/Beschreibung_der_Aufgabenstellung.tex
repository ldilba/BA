\section{Beschreibung der Aufgabenstellung}
Während meines Praktikums bei CONET habe ich an zwei Projekten gearbeitet. In der ersten Woche wurden alle technischen Rahmenbedingungen geschaffen. In dieser Zeit arbeitete ich mich hauptsächlich in das Web-Frontend Framework Angular ein. Die nächsten zwei Monate habe ich mit der Entwicklung des DashNET Projekts verbracht. Die letzten drei Wochen habe ich an der ePEP Software mitgearbeitet.

\subsection{DashNET}
Das Projekt mit dem Namen DashNET beschreibt ein modulares webbasiertes Dashboard. Die Hauptaufgabe war es, eine Softwarearchitektur zu entwerfen und diese prototypisch anhand eines Dashboards zu implementieren. Da Dashboards Daten aus unterschiedlichsten Quellen beziehen und die gesammelten Daten sehr individuell dargestellt werden, musste eine Architektur entwickelt werden, die diese Modularität unterstützt. Laut persönlicher Erfahrungswerte verschiedener CONET Mitarbeiter, kommt es während der Laufzeit eines Programmes, welches Daten von Drittanbietern bezieht, häufiger dazu, dass sich die Schnittstellen verändern. Dies ist teilweise schwer zu bemerken. Anpassungen an unterschiedlichen Stellen im Code sind sehr zeitaufwändig und daher mit hohen Kosten verbunden. DashNET soll diese Problematik lösen. Meine Aufgabe war die Entwicklung des Backend-Servers. 

\subsection{ePEP}

Das Programm ePEP ist ein sich bereits im Einsatz befindendes Organisationswerkzeug für die Personaleinsatzplanung. Innerhalb dieses Programms können Mitarbeiter, Teams und Projekte angelegt werden. Mitarbeiter werden anschließend Projekten zugeordnet, in denen sie mitwirken sollen. Da die Arbeitsstunden in den Stammdaten der Mitarbeiter eingepflegt werden können, kann eine Arbeitsauslastung sämtlicher zugeteilter Mitarbeiter berechnet werden. Dies ermöglicht eine schnelle Kapazitätsplanung und schafft einen Überblick, welcher Mitarbeiter an welchem Projekt arbeitet. Meine Aufgaben an der Software bestanden hauptsächlich aus Featureerweiterungen und kleineren Fehlerbehebungen. 