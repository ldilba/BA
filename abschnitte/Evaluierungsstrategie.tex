\section{Evaluierungsstrategie}
Die Evaluierung der innerhalb der Bachelorarbeit entwickelten Schnittstelle wird zunächst mit dem Lastenheft durchgeführt. Hierbei wird festgehalten, welche der Funktionalitäten implementiert wurden und welche noch nicht entwickelt werden konnten. Anhand der noch fehlenden Features kann auch eine Zukunftsprognose für die Schnittstelle entworfen werden. Nachdem die interne Evaluierung abgeschlossen ist, werden Probanden gebeten, eine testweise Implementation einer KI mithilfe der Schnittstelle durchzuführen. Dazu werden Daten und eine künstliche Intelligenz zur Datenverarbeitung bereitgestellt. Anschließend werden die Probanden mithilfe des System Usability Scale Fragebogens \footcite[System Usability Scale (SUS)]{Sus} nach einer Einschätzung der Bedienbarkeit der Software befragt. Dazu werden mehrere Fragen definiert, die ein Proband auf einer Skala von \glqq Strongly Agree\grqq{} bis \glqq Strongly Disagree\grqq{} in fünf Abstufungen beantworten kann.