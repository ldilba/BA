\section{Einleitung}
text
\subsection{Motivation und Hintergrund}
Jeden Tag werden riesige Mengen an Daten produziert. Im Jahr 2020 wurden weltweit
64,2 Zettabyte produziert.1 Dies entspricht 64.000.000.000.000 Gigabyte. Doch aus reinen
Daten kann nicht direkt Wissen abgeleitet werden. Mithilfe von Datenmanagement
und Datenanalyse wird versucht, die Daten soweit aufzubereiten, dass sie durch Menschen
und Computer ausgewertet werden können. Je nach Datenmenge und Abweichung
der Daten untereinander kann dies ein aufwändiger, langwieriger und damit teurer Prozess
sein. Bei größerer Komplexität oder Menge der Daten wird es für Menschen schwerer,
Zusammenhänge, Abweichungen und Auffälligkeiten zu erkennen. Dies liegt unter
anderem daran, dass Muster von neu erfassten Daten nur aus Erinnerungen aus dem
Kurzzeitgedächtnis abgeleitet werden können.2 Um dieses Problem zu lösen, wurden
Algorithmen entwickelt, die mit großen Datenmengen trainiert werden können um allgemeine
Aussagen über die eingegebenen Daten treffen zu können. Je nach Datenquelle
und Art der Aussage, die über diese Daten getroffen werden soll, werden unterschiedliche
Algorithmen aus dem Bereich der künstlichen Intelligenz benötigt. Bei dem Ansatz, eine
bestimmte Art von Datenquelle an eine KI anzubinden, entsteht eine feste Verdrahtung
zwischen dem Datenerhebungsalgorithmus und dem KI-gestützten Datenverarbeitungsalgorithmus.
Sollte sich entweder die Datenerhebung oder die Auswertung verändern,
muss in der Regel der gesamte Prozess überarbeitet werden. Dies kann nur von jemandem
durchgeführt werden, der sich mit den Daten, der eingesetzten künstlichen Intelligenz
und den dazu programmierten Schnittstellen auskennt.

\subsection{Problemstellung}
Das Thema der Bachelorarbeit soll die Entwicklung und Erarbeitung einer Methode sein,
mit der die durch die Datenerhebung gesammelten Daten leichter an die KI-gestützte
Datenverarbeitung angeschlossen werden können. Jede künstliche Intelligenz braucht
als Input Daten in einem bestimmten Datenformat. Dieses kann sich von Algorithmus zu
Algorithmus ändern. KI-basierte Textanalysealgorithmen wie der von Google entwickelte
BERT Algorithmus (Bidirectional Encoder Representations from Transformers) benötigen
reinen ASCII-Text als Input. Ein Entwickler, der eine KI mit gesammelten Daten benutzen
möchte, muss diese Daten vorher genau auf das Format bringen, welches die KI benötigt.
Sollte die KI oder der Datenerhebungsalgorithmus ausgetauscht werden, muss der Entwickler
darauf achten, dass die Daten auch weiterhin kompatibel sind und das gewollte
Ergebnis liefern. Die Frage ist demnach: „Wie kann ein Entwickler nach Einrichtung der
KI die Daten austauschen ohne dabei den gesamten Anschluss neu programmieren zu
müssen?“ Ebenso ist die andere Richtung eine zentrale Frage in der Bachelorarbeit. „Wie
kann ein Entwickler eine bereits angeschlossene KI mit einer anderen austauschen, ohne
die Daten verändern zu müssen?“
\subsection{Zielsetzung}
Im Rahmen der Bachelorarbeit wird ein Konzept entwickelt, welches es ermöglicht, die
Datenerhebung und Datenverarbeitung mithilfe einer Middleware dynamisch auszutauschen.
Die Betreiber der KI sollen lediglich eine Konfigurationsdatei anpassen müssen,
um die Daten für die KI vorzubereiten. Ziel der Bachelorarbeit wird es sein, eine Middleware
zu entwickeln, welche mithilfe dieser Konfigurationsdatei, Daten aus einem Datenerhebungsalgorithmus
mittels einer REST-API Schnittstelle annimmt und diese automatisch
für eine vorgesehene KI präpariert. Im Anschluss soll das Programm mit den vorbereiteten
Daten eine Anfrage an die KI senden und das Ergebnis dem Benutzer wieder zurücksenden.
Dieses Konzept wird beispielhaft an einem Textanalysealgorithmus, der eine
semantische Suche innerhalb eines Textes ermöglicht, implementiert.
Wichtig bei der Entwicklung ist es, dass sowohl Datenerhebung als auch Datenverarbeitung
modular entwickelt werden. Da die Middleware als REST-API entwickelt wird, können
zukünftige Entwickler die Schnittstelle nutzen, auch wenn sich die Daten oder die KIs
verändern sollten.
Die REST-API wird mithilfe des Python Web-Frameworks Flask entwickelt. Dies bietet
eine einfache und schnelle Möglichkeit einen Backend-Server aufzusetzen. Des weiteren
ist Python eine umfangreiche und einfach zu benutzende Programmiersprache für
Datenverarbeitung.
Die Konfiguration der Datenverarbeitung wird mit dem Web-Frontend Framework Angular
entwickelt. Dort soll der Entwickler eine Schritt für Schritt Anleitung für die Konfiguration
der Middleware präsentiert bekommen, in der er die Art der Daten, deren Format und die
anzuschließende KI auswählen kann.
Mithilfe dieser Middleware soll sich der Arbeitsaufwand verringern, der durch die Anbindung
neuer KIs entsteht. Damit können erfahrene Entwickler entlastet werden und Personalkosten
gespart werden. Ebenso können schnell und ohne großen Mehraufwand alternative
Aufbereitungen der Daten oder neue KIs getestet werden.
\subsection{Stand der Forschung}
text?
\subsection{Vorgehen}
text?