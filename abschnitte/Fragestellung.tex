\section{Fragestellung}
Das Thema der Bachelorarbeit soll die Entwicklung und Erarbeitung einer Methode sein, mit der die durch die Datenerhebung gesammelten Daten leichter an die KI-gestützte Datenverarbeitung angeschlossen werden können. Jede künstliche Intelligenz braucht als Input Daten in einem bestimmten Datenformat. Dieses kann sich von Algorithmus zu Algorithmus ändern. Beispielsweise brauchen bestimme Spracherkennungsalgorithmen reinen ASCII-Text oder Bilderkennungsalgorithmen JPEG Dateien mit einer festgelegten Auflösung. Ein Entwickler, der eine KI mit gesammelten Daten trainieren möchte, muss diese Daten vorher genau auf das Format bringen, welches die KI benötigt. Sollte die KI oder der Datenerhebungsalgorithmus ausgetauscht werden, muss der Entwickler darauf achten, dass die produzierten Daten auch weiterhin kompatibel sind und das gewollte Ergebnis liefern. Die Frage ist demnach: \glqq Wie kann ein Entwickler nach Einrichtung der KI die Daten austauschen ohne dabei den gesamten Anschluss neu programmieren zu müssen?\grqq{} Ebenso ist die andere Richtung eine zentrale Frage in der Bachelorarbeit.\glqq Wie kann ein Entwickler eine bereits angeschlossene KI mit einer anderen austauschen, ohne die Daten verändern zu müssen?\grqq{}