\section{Methodik}
Die Schnittstelle für die Anbindung von austauschbaren Datenquellen an KI-Algorithmen wurde auf Grundlage eines modifizierten Wasserfallmodells entwickelt. Das Wasserfallmodell ist ein im Jahre 1970 von Winston W. Royce erstmals beschriebener Designprozess zur Entwicklung von Softwareanwendungen.\footcite{model2015waterfall} Das Modell beschreibt einen sequenziellen Prozess der Softwareentwicklung. Dieser Prozess besteht aus fünf Unterpunkte: \footcite{adenowo2013software}

\begin{enumerate}
\item Requirement Analysis,
\item Design,
\item Implementation,
\item Testing,
\item Operation and Maintenance.
\end{enumerate}

Da das Wasserfallmodell einen sequenziellen Prozess beschreibt, muss jeder Schritt vollständig abgeschlossen sein, bevor der nächste Schritt gestartet werden kann. Rückschritte sind in diesem Modell nicht vorgesehen. Ähnlich wie bei einem Wasserfall kann der Projektfluss nur von oben nach unten ablaufen.\footcite{PorathRon2020ICuI}

Das Wasserfallmodell wurde für den Prototyp modifiziert. Nicht übernommen wurde der vierte Schritt \glqq Testing\grqq{} und der fünfte Schritt \glqq Operation and Maintenance\grqq{}. Für den Prototyp wurden stattdessen die Schritte \glqq Evaluation\grqq{} und \glqq Anpassen\grqq{} implementiert.

Das angepasste Wasserfallmodell mit den dazugehörigen Kapiteln, in denen die einzelnen Schritte umgesetzt werden, ist in Tabelle 2 aufgeführt:

\begin{table}[H]
\centering
\begin{tabular}{c|l|l}
\textbf{Schritt} & \textbf{Bezeichnung} & \textbf{Kapitel}\\
\hline
1 & Anforderungen definieren & 4.1 Anforderungen\\
2 & Konzept entwickeln & 4.2 Konzeption\\
3 & Prototypische Implementierung & 4.3 Prototypische Umsetzung\\
4 & Evaluation & 5 Evaluation\\
5 & Anpassen & 6.3 Ausblick\\
\end{tabular}
\caption{Schritte des modifizierten Wasserfallmodells}
\end{table}

Für die Schnittstelle wurde sich gegen ein alternatives Projektmanagementmodell wie Scrum entschieden. Scrum bietet viele Vorteile bei Entwicklerteams, die gemeinsam an einer Software arbeiten. Laut der Untersuchung von Harwardt wirkt Scrum einigen Risiken der Softwareentwicklung entgegen. Es ist innerhalb des Entwicklungszeitraums möglich Änderungen vorzunehmen und diese regelmäßig mit Kunden abzusprechen. Ebenso steigert es die Mitarbeiterzufriedenheit und Motivation im Gegensatz zur Arbeit mit dem Wasserfallmodell.\footcite{harwardt2012wasserfallmodell} Trotz dieser Vorteile von Scrum wurde für die Entwicklung der Schnittstelle das Wasserfallmodell gewählt. Für die Schnittstelle, gibt es keine Kunden. Es waren keine weiteren Entwickler an der Software beteiligt. Ein Framework zur regelmäßigen Absprache mit Kunden und anderen Mitarbeitern ist demnach nicht notwendig. Der Entwicklungszeitraum betrug sechs Wochen. Eine Aufteilung dieser Zeit in mehrere Sprint-Zyklen, wie es mit dem Scrum-Modell notwendig wäre, bietet ebenfalls keinen Mehrwert. Scrum hat im Vergleich zum Wasserfallmodell einen größeren Verwaltungsaufwand zur Folge, da die zu entwickelnden Funktionalitäten der Schnittstelle in User-Stories formuliert und der Arbeitsaufwand mit Story-Points abgeschätzt werden muss.\footcite{wirdemann2022scrum}

\subsection{Anforderungen definieren}
Im ersten Schritt des Wasserfallmodells werden die Anforderungen an die zu entwickelnde Software erhoben nach den Vorgaben der IEC 62304 Kapitel 5.2\footcite{daniel2018anforderungen} zur Anforderungserhebung für Softwaresysteme. Die Anforderungserhebung findet in Zusammenarbeit mit der CONET statt. Die Anforderungen werden mit einem Experten aus dem Bereich der Softwareentwicklung mit dem Spezialgebiet KI ausgearbeitet. CONET stellt die Ideenbeschreibung für die Schnittstelle zur Verfügung. Auf Basis dieser Beschreibung wird eine Anforderungsliste entworfen, die die folgenden fünf Punkte der IEC 62304 abdeckt:

\begin{itemize}
\item Benutzerschnittstellen,
\item die GUI,
\item Verhalten des Systems auf Benutzeraktionen im Normal- und Fehlerfall,
\item die Geschwindigkeit, mit der diese Reaktionen geschehen,
\item auf welchen Umgebungen, sich die Software installieren lassen soll.
\end{itemize}

In Kapitel 4.1 ist eine schriftliche Ausarbeitung der Liste dokumentiert.

\subsection{Konzept entwickeln}
Im zweiten Schritt des Wasserfallmodells wird das Konzept für die Software erstellt. Die zuvor definierten Anforderungen bieten dafür eine Leitlinie. Für die verwendeten Technologien wird ein Konzept entworfen. Innerhalb dieses Konzepts wird die Architektur der Software beschrieben. In Kapitel 4.2.1 \glqq Softwarearchitektur\grqq{} ist das Zusammenspiel und die Abhängigkeiten der einzelnen Bestandteile des Systems aufgeführt. Bei der zu entwickelnden Anwendung handelt es sich um eine Full-Stack-Architektur, die sowohl ein Frontend als auch ein Backend mit den dazugehörigen Komponenten beinhaltet. Daher ist die Planung der Architektur für den Entwicklungsprozess entscheidend.\footcite{taivalsaari2021full} Änderungen der Architektur nach dem Aufsetzen der einzelnen Systeme sind aufwändig und zeitintensiv.

Das Konzept beinhaltet neben der Planung der Architektur einen Programmablaufplan. Dieser Plan ist wird in Kapitel 4.2.2 \glqq Programmablauf\grqq{} beschrieben. Innerhalb des Programmablaufplans wird die Nutzerinteraktion mit dem System und die dadurch ausgelösten Ereignisse im Backend erläutert. In diesem Schritt werden Rand- und Fehlerfälle nicht behandelt.

Zur Visualisierung des Konzepts werden Mockups für die Website entworfen. Da die Ereignisse innerhalb des Prototyps durch einen Anwender ausgelöst werden, ist die Darstellung der Eingabefelder, Buttons und Ergebnisanzeigen zum Verständnis des Programmablaufs hilfreich. Mockups verbessern den Prozess der Erhebung und Bestätigung der Anforderungen.\footcite{rivero2010mockups}

\subsection{Prototypische Implementierung}
Der dritte Schritt des Wasserfallmodells umfasst die prototypische Umsetzung der Schnittstelle. Die dazu verwendeten Werkzeuge werden in Kapitel 2.5 \glqq Werkzeuge\grqq{} beschrieben. Bei der Auswahl der Werkzeuge für die Erstellung der Schnittstelle wird darauf geachtet, moderne und in der Praxis genutzte Frameworks und Technologien zu verwenden. Die Auswahl basiert auf Expertenmeinungen von CONET, eigener Recherche und Technologien, die bei CONET im Einsatz sind. Das zuvor definierte Konzept dient als Anleitung für den Entwurf und die Implementierung der Softwarearchitektur. Die Mockups werden als visuelle Referenz für das Design des Frontends verwendet. Der Artikel von Tibshirani zur Implementierung einer Textähnlichkeitssuche in Elasticsearch dient als Grundlage für den KI-Service.\footcite{tibshirani2019ki} In den KI-Service wird eine implementierte Version des BERT-Modells von Tibshirani integriert.\footcite{tibshirani2020github} Die Implementierung der Software wird in Kapitel 4.3 \glqq Prototypische Umsetzung\grqq{} beschrieben.

\subsection{Umsetzung evaluieren}
Die prototypischen Implementierung wird mit den zu Beginn der Arbeit erhobenen Anforderungen in regelmäßigen Code-Reviews verglichen. Die Durchführung der Code-Reviews findet mit einem Experten von CONET statt. Innerhalb dieser Code-Reviews liegt der Fokus auf der Fertigstellung des Prototyps sowie der Qualität des Codes. Die Art der Implementierung der in den Anforderungen erhobenen Features ist ebenfalls Bestandteil des Code-Reviews. Des Weiteren wird auf die Modularität, Skalierbarkeit und Performance des Codes geachtet.

Die Ergebnisse der Code-Reviews sind in Kapitel 5 \glqq Evaluation\grqq{} festgehalten. 

\subsection{Anpassen}
Im letzten Schritt des Wasserfallmodells \glqq Anpassen\grqq{} wird der Prototyp Mitarbeitenden und Experten der CONET im Rahmen einer vorgestellt. Das nach der Präsentation gesammelte Feedback wird in Kapitel 6.3 \glqq Ausblick\grqq{} aufgeführt. 