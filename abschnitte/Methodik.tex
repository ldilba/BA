\section{Methodik}
Lineare Projektvorgehensmethode, bei welcher in jeder Projektphase definierte Ziele und
Tätigkeiten fertiggestellt werden, bevor die nächste Phase startet. Dies erlaubt die ein-
fachere und bewährte Durchführung eines Projekts, bietet jedoch im Gegensatz zu agilen
Projektmethoden weniger Änderungsmöglichkeiten nach dem Beginn der Entwicklungs-
phase des Projekts

\footcite{PorathRon2020ICuI}
\footcite{harwardt2012wasserfallmodell}

\subsection{Anforderungen definieren}
Im ersten Schritt werden alle Anforderungen an die zu entwickelnde Software erhoben.
Die Anforderungserhebung wird in Zusammenarbeit mit der Conet gemacht. In einem
Meeting mit Experten sollen mittels Brainstorming alle Aspekte der Middleware durchgegangen
werden und dabei alle geplanten Features, erwünschten Ergebnisse sowie
erforderlichen Dokumentationen festgehalten werden.

\subsection{Konzept entwickeln}
Im zweiten Schritt des Wasserfallmodells wird das Konzept für die Software erstellt. Die
zuvor definierten Anforderungen bieten dafür eine Leitlinie. In diesem Schritt wird bereits
die zu implementierende KI analysiert, um festzustellen, welche Ein- und Ausgaben diese
fordert. Anhand den Anforderungen der Experten und den Anforderungen der KI kann eine
Softwarearchitektur und ein Programmablauf entwickelt werden. Ebenso werden hier alle
verwendeten Technologien und benötigten Programme festgehalten.

\subsection{Prototypisch umsetzen}
Die anschließend erfolgende prototypische Umsetzung wird den Großteil der Bachelorarbeit
ausmachen. Dort wird die Schnittstelle zwischen Datenerhebungs- und Datenverarbeitungsalgorithmen
entwickelt und dokumentiert. Ebenso wird die Benutzerschnittstelle
für die Einrichtung der Middleware entwickelt. Das zuvor erarbeitete Konzept kann hier
als Leitlinie für die zu entwickelnde Middleware verwendet werden.

\subsection{Umsetzung evaluieren}
Die Umsetzung wird nach der prototypischen Implementierung mit den zu Beginn der
Arbeit erhobenen Anforderungen verglichen. Nach Abschluss der prototypischen Umsetzung
wird eine Präsentation der Middleware in einer Expertenrunde gehalten. Anschließend
wird mit den Experten ein Architektur- und Codereview des Backends durchgeführt.
Währenddessen wird auch die Bedienbarkeit und Verständlichkeit der Middleware evaluiert. Dazu werden beispielhafte Daten zur Implementation vorbereitet und eine vorher
vorbereitete KI zur Auswertung der Daten bereitgestellt.

\subsection{Anpassen}
Zuletzt wird in der „Anpassen“ Phase das Feedback der Probanden genutzt, um eventuelle
Schwachstellen, Unklarheiten oder Fehler der Software zu beheben. Alle weiteren
gesammelten Informationen werden in der Bachelorarbeit im Fazit und im Ausblick behandelt.