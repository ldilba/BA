\section{Fazit und Ausblick}
\subsection{Fazit}
Für diese Bachelorarbeit wurde eine Architektur für eine Schnittstelle zur Anbindung von Daten an KI-Algorithmen mithilfe des Wasserfallmodells  entwickelt. Der Prototyp besteht aus drei Hauptkomponenten. 

Die erste Komponente ist das mit Angular umgesetzte Frontend. Das Frontend dient als Benutzerschnittstelle für die Eingabe von Anfragen an eine KI. Des Weiteren dient das Frontend zur Eingabe einer Transformationsanleitung, mit der die Eingabe automatisiert auf das von der KI benötigte Format umgewandelt werden kann. Der letzte Einsatzzweck des Frontends ist die Auswahl des KI-Services und der Anzeige der durch die KI produzierten Ergebnisse. Für die Entwicklung einer Website wurde ein Konzept gezeigt und implementiert, welches einen sequenziellen Prozess abbilden kann. Die Website unterstützt das dynamische Hinzufügen und Entfernen von KI-Services, ohne dass der Code der Website angepasst werden muss. Die Website implementiert ebenfalls ein System zum asynchronen Laden von Ergebnissen aus den KI-Services. Während der Entwicklung wurde auf Modularität und Wiederverwertbarkeit von Code geachtet, wodurch, das das System auch für Anfragen genutzt werden kann, die keine KI als Ziel haben. 

Die zweite Komponente ist die in Python mit dem Package Flask entwickelte REST-API. Innerhalb der API wurden mehrere Routen definiert, die zur Interaktion mit dem Backend genutzt werden können. Für das Backend wurde im Rahmen der Bachelorarbeit ein Konzept entworfen, Fehler zur Laufzeit abzufangen und behandeln zu können. Innerhalb der API wurde ein System zum Loggen von Systemereignissen konzipiert. Die API wurde nach dem Grundsatz einer Microservice-Architektur entworfen, die eine horizontale Skalierung ermöglicht. Sollte es bei der Nutzung des Systems zu einer hohen Auslastung durch eine große Anzahl an Nutzern kommen, so können beliebig viele Instanzen der API dazugeschaltet werden.

Die dritte Komponente ist die mit RabbitMQ realisierte Kommunikation zu den KI-Services. RabbitMQ unterstützt die entworfene Microservice-Architektur als Nachrichtenaustauschdienst zwischen dem Backend und den Services. Dieser Dienst ermöglicht ist eine asynchrone Kommunikation zwischen den beiden Kommunikationspartnern. RabbitMQ erweitert die im Backend implementierte Skalierbarkeit. Eine horizontale Skalierung der KI-Services ist dadurch erreicht worden.

Die entwickelte Schnittstelle zur Anbindung von austauschbaren Datenquellen an KI-Algorithmen wurde entworfen und in eine praxistaugliche Architektur integriert, die für den Einsatz in einer Produktivumgebung ausgelegt ist. 

\subsection{Ausblick}
Die Bachelorarbeit stellt die Grundlage für die Umsetzung der Projektidee von CONET dar, ein Trendscouting-System zu entwickeln. Dieses System soll zunächst mehrere Datenquellen, wie wissenschaftliche Paper, Auszüge aus Büchern, Webseiten und Nachrichtenfeeds sammeln. Die gesammelten Daten sollen dazu genutzt werden, erkennen zu können, welche Themen im Bereich der IT in den nächsten Jahren relevant werden. Dazu benötigt es eine KI, die textbasierte Daten auf semantische Ähnlichkeit vergleicht. Die grundlegende Funktionalität dazu wurde mit der Eingabe einer Query, dessen Transformation über eine weitere Eingabe im Frontend und die KI-basierte Auswertung bereitgestellt.

Die für den den Einsatz in einer Produktivumgebung noch zu verändernden Aspekte der Software wurden in einem Review des Prototyps mit mehreren Mitarbeitern von CONET festgehalten. Im Prototyp ist das System zur Nutzung mehrerer Anwender nicht implementiert. Im Backend ist die Funktionalität zur Unterscheidung von Nutzern durch eine UUID, die über JWT transportiert wird, bereits implementiert. Im Frontend muss ein System zur Nutzung einer UUID noch implementiert werden, bevor die Architektur in einer Produktivumgebung deployed werden kann.  

Im Review wurde für die künftige Weiterentwicklung wurden Erweiterung der Funktionalität mehrfach erwünscht. Für den Prototyp ist die Eingabe einer Query durch den Nutzer ausreichend. Ein etwaiges Trendscouting-System benötigt ein automatisiertes System zur Eingabe von Queries. Das System kann um die Möglichkeit der Eingabe vollständiger Dokumente erweitert werden. Die Dokumente sollten sowohl für die Anfrage an die KI nutzbar sein als auch den Datenbestand der Elasticsearch erweitern. 

Der Output der im Prototyp implementierten KI beinhaltet einen Titel und eine Beschreibung. Der Output lässt sich dahingehend erweitern, dass nicht nur textbasierte Ergebnisse angezeigt werden können, sondern auch ganze Dokumente in die Ergebnisliste geladen werden können. Es sollte ebenfalls möglich sein, diese Dokumente anschließend herunterzuladen.

Mit dem Prototyp wurde eine Grundlage für die modulare Anbindung von Daten an KI-Algorithmen geschaffen. Die Softwarearchitektur ermöglicht die Weiterentwicklung für viele Anwendungszwecke. 
