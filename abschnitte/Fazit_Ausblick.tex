\section{Fazit und Ausblick}
\subsection{Fazit}
Für diese Bachelorarbeit wurde mithilfe des Wasserfallmodells eine Architektur für eine Schnittstelle zur Anbindung von Daten an KI-Algorithmen entwickelt. Der aus der Entwicklung entstandene Prototyp besteht aus drei Hauptkomponenten. 

Die erste Komponente ist das mit Angular umgesetzte Frontend. Das Frontend dient als Benutzerschnittstelle für die Eingabe von Anfragen an eine KI. Des Weiteren dient das Frontend zur Eingabe einer Transformationsanleitung, mit der die Eingabe automatisiert auf das von der KI benötigte Format umgewandelt werden kann. Der letzte Einsatzzweck des Frontends ist die Auswahl des KI-Services und der Anzeige der durch die KI produzierten Ergebnisse. Für die Entwicklung einer Website wurde ein Konzept gezeigt und implementiert, welches einen sequenziellen Prozess abbilden kann. Die Website unterstützt das dynamische Hinzufügen und Entfernen von KI-Services, ohne dass der Code der Website angepasst werden muss. Des Weiteren implementiert die Website ein System zum asynchronen Laden von Ergebnissen aus den KI-Services. Da während der Entwicklung auf Modularität und Wiederverwertbarkeit von Code geachtet wurde, kann das System auch für Anfragen genutzt werden, die keine künstliche Intelligenz als Ziel haben. 

Die zweite Komponente ist die in Python mit dem Package Flask entwickelte REST-API. Innerhalb der API wurden mehrere Routen definiert, die zur Interaktion mit dem Backend genutzt werden können. Für das Backend wurde im Rahmen der Bachelorarbeit ein Konzept entworfen, Fehler zur Laufzeit abzufangen und behandeln zu können. Des Weiteren wurde ein System zum Loggen von Systemereignissen konzipiert. Die API wurde nach dem Grundsatz einer Microservice-Architektur entworfen, was eine horizontale Skalierung ermöglicht. Sollte es bei der Nutzung des Systems zu einer hohen Auslastung durch eine große Anzahl an Nutzern kommen, so können beliebig viele Instanzen der API dazugeschaltet werden.

Die dritte Komponente ist die mit RabbitMQ realisierte Kommunikation zu den KI-Services. RabbitMQ unterstützt die entworfene Microservice-Architektur als Nachrichtenaustauschdienst zwischen dem Backend und den Services. Durch diesen genutzten Dienst ist eine asynchrone Kommunikation zwischen den beiden Kommunikationspartnern ermöglicht worden. Die im Backend implementierte Skalierbarkeit wird durch RabbitMQ erweitert. Dadurch wurde die horizontale Skalierung der KI-Services erreicht.

Die entwickelte Schnittstelle zur Anbindung von austauschbaren Datenquellen an KI-Algorithmen wurde entworfen und in eine praxistaugliche Architektur integriert, die für den Einsatz in einer Produktivumgebung ausgelegt ist. 

\subsection{Ausblick}
Die im Rahmen der Bachelorarbeit entwickelte Softwarearchitektur bietet in vielen Bereichen Raum zur Weiterentwicklung. Die Bachelorarbeit stellt die Grundlage für die Umsetzung der Projektidee der CONET Solutions GmbH, ein Trendscouting-System zu entwickeln, dar. Dieses System soll zunächst mehrere Datenquellen, wie wissenschaftliche Paper, Auszüge aus Büchern, Webseiten und Nachrichtenfeeds sammeln. Die gesammelten Daten sollen dazu genutzt werden, erkennen zu können, welche Themen im Bereich der IT in den nächsten Jahren relevant werden. Dazu benötigt es eine Künstliche Intelligenz, die textbasierte Daten auf semantische Ähnlichkeit vergleicht. Die grundlegende Funktionalität dazu wurde mit der Eingabe einer Query, dessen Transformation über eine weitere Eingabe im Frontend und die KI-basierte Auswertung bereitgestellt.

Die für den den Einsatz in einer Produktivumgebung noch zu verändernden Aspekte der Software wurden in einem Review des Prototyps mit mehreren Mitarbeitern von CONET festgehalten. Aktuell gibt es im Prototyp keine Möglichkeit das System mit mehreren Nutzern gleichzeitig zu verwendet. Im Backend ist die Funktionalität zur Unterscheidung von Nutzern durch eine UUID, die über JWT transportiert wird, bereits implementiert. Im Frontend muss ein System zur Nutzung einer individuellen ID jedoch noch implementiert werden, bevor die Architektur deployed werden kann.  

Aus dem Review kristallisierten sich auch mehrere Wünsche zur Funktionalitätserweiterung heraus. Die Eingabe einer Query durch den ist für den Prototyp ausreichend. Für ein Trendscouting-System wird an dieser Stelle jedoch ein automatisiertes System zur Eingabe von Queries benötigt. Des Weiteren kann das System dahingehend erweitert werden, dass als Eingabe vollständige Dokumente genutzt werden können. Diese sollten sowohl für die Anfrage an die KI nutzbar sein als auch den Datenbestand der Elasticsearch erweitern. 

Der aktuelle Output der KI beinhaltet einen Titel und eine Beschreibung. Der Output lässt sich dahingehend erweitern, dass nicht nur textbasierte Ergebnisse angezeigt werden können, sondern auch ganze Dokumente in die Ergebnisliste geladen werden können. Es sollte ebenfalls möglich sein, diese Dokumente anschließend herunterzuladen.

Dieses Kapitel zeigt, dass die konzipierte Schnittstelle noch viele Möglichkeiten zur Weiterentwicklung bietet. Der Prototyp legt die Grundlage für die modulare Anbindung von Daten an KI-Algorithmen. Für ein Trendscouting-System und weitere Anwendungszwecke dieser Architektur bedarf es allerdings noch weitere Entwicklungszeit.
